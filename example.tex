\documentclass{mlareport}

% For colored 'citation needed' warnings. Recommended but not required.
\usepackage{xcolor}

% For filler text. Not required.
\usepackage{lipsum}

% Define data for the title and heading.
\title{Example Document for the ``\texttt{mlareport}'' Document Class}
\author{Kian}{Kasad} % Requires both a first and a last name
\instructor{Prof. Professor}
\course{\LaTeX{} 101}
\date{27 Apr. 2022}

\subject{A subject that will only appear in the PDF metadata,
	not on the typeset document}

\begin{document}

% Print the heading and the title. These are technically not required,
% but then you'd have to do this manually.
% These typeset the data given above.
\makeheading
\maketitle

% Filler text. Not required.
\lipsum[1-3]

% Footnote text definition. The label "fn disclaim" is given in order to
% reference this later for inclusion in the text. This way, the footnote
% doesn't interrupt the paragraph and you don't rely on footnote numbers
% that could change later.
\fntext{fn disclaim}{
	Footnote text must be defined using
	\texttt{\backslash fntext}
	before placing the footnote in
	the document text using
	\texttt{\backslash fn}.
}

This is a very short paragraph to display the labeled footnotes
functionality.\fn{fn disclaim} This sentence contains a fact that is not
cited yet \needcite.

% Filler text. Not required.
\lipsum[4-7]

% Used to create the 'Works Cited' page
\begin{workscited}

% Use '\wcentry' to start a new source listing
\wcentry
	Wikipedia contributors.
	``LaTeX.''
	\textit{Wikipedia, The Free Encyclopedia}.
	Wikipedia, The Free Encyclopedia,
	26 Apr. 2022.
	\url{en.wikipedia.org/w/index.php?title=LaTeX&oldid=1084788103}.
	Accessed 28 Apr. 2022.

\wcentry
	Wikipedia contributors.
	``TeX Live.''
	\textit{Wikipedia, The Free Encyclopedia}.
	Wikipedia, The Free Encyclopedia,
	11 Apr. 2022.
	\url{en.wikipedia.org/w/index.php?title=TeX_Live&oldid=1082115315}.
	Accessed 28 Apr. 2022.

\end{workscited}

\end{document}

% vim: ft=tex ts=4 sw=4 noet tw=72
